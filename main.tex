\documentclass[a4paper]{article}

\usepackage[english]{babel}
\usepackage[utf8x]{inputenc}
\usepackage[affil-it]{authblk}
\usepackage{amsmath,amsthm}
\usepackage{amssymb}
\usepackage{enumitem}


\title{Complex Analysis: Important Results}
\author{Alexander Rambasek}
\affil[]{MATH 324: Introduction to Complex Analysis}
\affil[]{Case Western Reserve University}

\affil[]{}

\theoremstyle{definition}

\newtheorem{definition}{Definition}
\newtheorem{theorem}{Theorem}
\newtheorem{lemma}{Lemma}
\newtheorem{corollary}{Corollary}

\begin{document}
	\maketitle
	
	%-------DEFINITIONS-------%
	
	\begin{definition}
		\emph{(Multiplication in $\mathbb{C}$)}
		Let $z$, $w$ $\in \mathbb{C}$. The product $\textbf{zw}$ is
		the complex number whose polar coordinates are $r = |z||w|$ and
		$\theta = \alpha + \beta$, where $\alpha$ and $\beta$ are the polar angles of $z$ and $w$. Equivalently, if $z = x + \textit{i}y$ and $w = p + \textit{i}q$, then $zw = xp - yq + \textit{i}(xq + yp)$.
	\end{definition}
	
	\begin{definition}
		Let $z = x + \textit{i}y$ $\in \mathbb{C}$. The \textbf{modulus}, or \textbf{absolute value}, of $z$ is denoted $|z|$, and is defined as follows:
		\begin{center}
			$|z| = \sqrt{x^2 + y^2}$.
		\end{center}
	\end{definition}
	
	\begin{definition}
		Let $z = x + \textit{i}y$ $\in \mathbb{C}$. The \textbf{complex conjugate} of z is denoted $\overline{z}$, and is defined as follows:
		\begin{center}
			$\overline{z} = x - \textit{i}y$.
		\end{center}
	\end{definition}
	
	\begin{definition}
		Let $z$ $\in \mathbb{C}$, $z \neq 0$. The polar angle of $z$ is called the \textbf{argument}, and denoted arg($z$). The value of arg($z$) in the range $[-\pi, \pi)$ is called the \textbf{principle value}, and is denoted Arg(z).
	\end{definition}
	
	\begin{definition}
		\emph{(Circles)}
		The circle with center $c \in \mathbb{R}$ and radius $r \in \mathbb{R}$, $r > 0$ is parameterized by the equation $|z - c| = r$, $z$ $\in \mathbb{C}$.
	\end{definition}
	
	\begin{definition}
		\emph{(Lines)}
		Consider a line $L$. Let $c \in \mathbb{C}$ represent a point on the line, and let $d \in \mathbb{C}, d \neq 0$ represent the direction of the line. The \textbf{parametric equation} of $L$ is written as $z(t) = ct + d$, for $t \in \mathbb{R}$. The \textbf{intrinsic equation} of $L$ is written as $\Re{[(z - c)i\overline{d}]} = 0$. For a (nonvertical) straight line $y = mx + b$, $m$ and $b$ real, the equation can be formulated as $\Re{[(m+i)z + b]} = 0$. Moreover, for $Px + Qy = K$, we get $\Re{[z(P - iQ)]} = K$.
	\end{definition}
	
	\begin{definition}
		Let $p \in \mathbb{C}$. The \textbf{translation} $T_{p}$ is defined as $T_{p}(z) = z + p$.
	\end{definition}
	
	\begin{definition}
		Let $u = \cos{\theta} + i \sin{\theta}$. The \textbf{rotation} $R_{\theta}$ is defined as $R_{\theta}(z) = u \cdot z$ (counterclockwise rotation about the origin by angle $\theta$).
	\end{definition}
	
	\begin{definition}
		Let $z \in \mathbb{C}$, $a \in \mathbb{R}$, $a > 0$. The \textbf{scaling} $S_{a}$ is defined as $S_{a}(z) = a \cdot z$.
	\end{definition}
	
	\begin{definition}
		Let $z \in \mathbb{C}$, $b \in \mathbb{C}$. The \textbf{multiplication} operator $M_{b}$ is defined as $M_{b}(z) = b \cdot z$.
	\end{definition}
	
	\begin{definition}
		Let $z \in \mathbb{C}$. The \textbf{inversion} mapping, $J$, is defined as $J(z) = \frac{1}{z}$.
	\end{definition}
	
	\begin{definition}
		Let $p \in \mathbb{C}$ and $r > 0$. The \textbf{open ball} of radius $r$ about $p$ is the set $\{z \; \, | \; \, |z-p| < r\}$.
	\end{definition}
	
	\begin{definition}
		If $p \in S \subset \mathbb{C}$, then $p$ is an \textbf{interior point} of $S$ if $S$ contains some open ball about $p$.
	\end{definition}
	
	\begin{definition}
		$S \subset \mathbb{C}$ is an \textbf{open set} if every point of $S$ is an interior point of $S$.
	\end{definition}
	
	\begin{definition}
		$S \subset \mathbb{C}$ is \textbf{closed} if its complement $\mathbb{C} \setminus S$ is open.
	\end{definition}
	
	\begin{definition}
		$p \in \mathbb{C}$ is a \textbf{boundary point} of $S$ (denoted $p \in \partial S$) if it is neither interior to $S$ nor to its complement.
	\end{definition}
	
	\begin{definition}
		$S \subset \mathbb{C}$ is a \textbf{neighborhood} of $p \in \mathbb{C}$ if $p$ is an interior point of $S$.
	\end{definition}
	
	\begin{definition}
		$N$ is a \textbf{neighborhood of infinity} if $\{0\} \cup \{\frac{1}{z} \; \, | \; \, z \in N \}$.
	\end{definition}
	
	\begin{definition}
		\emph{(Connectedness and Domains)}
		Consider an open subset $S \subset \mathbb{C}$.
		\begin{enumerate}[label=\alph*.]
			\item $S \subset \mathbb{C}$ is \textbf{disconnected} if it is the disjoint union of two nonempty open sets.
			\item $S \subset \mathbb{C}$ is \textbf{connected} if it is not disconnected.
			\item $S$ is \textbf{polygonally connected} if for any $z$, $w \in S$ there is a finite sequence of points $p_{0}$, $p_{1}$, $... p_{n}$ with $z = p_{0}$, $w = p_{n}$, such that $S$ contains the line segment $p_{j-1}p_{j}$ for each $1 \leq j \leq n$.
		\end{enumerate}
	\end{definition}
	
	\begin{definition}
		A \textbf{domain} is an open, connected subset of $\mathbb{C}$.
	\end{definition}
	
	\begin{definition}
		A subset $S \subset \mathbb{C}$ is \textbf{convex} if for any $z, w \in S$, the line segment from $z$ to $w$ is in $S$.
	\end{definition}
	
	\begin{definition}
		A set $K$ is \textbf{compact} if, whenever there is a collection of open sets whose union contains $K$, then there is a finite sub-collection of the open sets that contains $K$.
	\end{definition}
	
	\begin{definition}
		A \textbf{function} $f : X \rightarrow Y$ is a rule assigning a single point $f(x) \in Y$ to each point $x \in X$.
	\end{definition}
	
	\begin{definition}
		The \textbf{image}, or \textbf{range}, of $f$ is the set $\{ f(x) \; \, | \; \, x \in X\} \subset Y$.
	\end{definition}
	
	\begin{definition}
		$\lim_{z \to p} f(z) = L$ means that $\forall \epsilon > 0$, $\exists \delta > 0$ such that $0 < |z - p| < \delta \implies |f(z) - L| < \epsilon$.
	\end{definition}
	
	\begin{definition}
		$\lim_{z \to \infty} f(z) = L$ means that $\forall \epsilon > 0$, $\exists r > 0$ such that $|z| > r \implies |f(z) - L| < \epsilon$.
	\end{definition}
	
	\begin{definition}
		A function $f$ is \textbf{continuous} at $p \in \mathbb{C}$ if $f$ is defined on a neighborhood of $p$ and if $\lim_{z \to p} f(z) = f(p)$.
	\end{definition}
	
	\begin{definition}
		If $f: X \rightarrow Y$ and $S \subset Y$, then the \textbf{pre-image} of $S$ under $f$, denoted $f^{-1}(S)$, is the set of all $x \in X$ such that $f(x) \in S$.
	\end{definition}
	
	\begin{definition}
		A \textbf{sequence} $f(n) = a(n) : \mathbb{N} \rightarrow \mathbb{C}$ converges to $L$ if $\lim_{n \to \infty} f(n) = L$.
	\end{definition}
	
	\begin{definition}
		A \textbf{series} in $\mathbb{C}$ is a sequence $s_{n} = \sum_{j=1}^{n} a_{j}$ of partial sums of another sequence $a_{n}$.
	\end{definition}
	
	\begin{definition}
		\emph{(Exponential)}
		For $z \in \mathbb{C}$, $e^{z} = e^{x + iy} = e^{x}e^{iy} = e^{x}[\cos{(y)} + i \sin{(y)}]$.
	\end{definition}
	
	\begin{definition}
		\emph{(Logarithm)}
		The \textbf{logarithm} function is a set-valued function defined on $\mathbb{C} \backslash \{0\}$ that sends a point $w$ to the set of all of its pre-images under the exponential function: $\log(w) = \ln|w| + i$ $\text{arg}(w)$.
	\end{definition}
	
	\begin{definition}
		Let $D = \mathbb{C} \; \backslash \; \{tc \; | \; t \geq 0\}$ for fixed $c \neq 0$. Let $\theta = \text{Arg}(z)$. There exists a single-valued \textbf{branch} of the logarithm continuous on all of $D$, defined as $\ln{|z|} + i a(z)$, where $a(z)$ is the value of $\text{arg}(z)$ lying in the range $[\theta, \theta + 2\pi)$.
	\end{definition}
	
	\begin{definition}
		The \textbf{principle value} of the logarithm is the single-valued function $\text{Log}(w)$ = $\ln|w| + i$ $\text{Arg}(w)$.
	\end{definition}
	
	\begin{definition}
		\emph{(Arbitrary Powers)}
		For $a$, $z \in \mathbb{C} \backslash \{0\}$, $a^{z}$ is defined to be the multi-valued function $e^{z \text{log}(a)}$.
	\end{definition}
	
	\begin{definition}
		\emph{(Hyperbolic Functions)}
		The \textbf{hyperbolic cosine} is defined as $\cosh{(z)} = \frac{e^{z} + e^{-z} }{2}$. The \textbf{hyperbolic sine} is defined as $\sinh{(z)} = \frac{e^{z} - e^{-z} }{2}$.
	\end{definition}
	
	\begin{definition}
		\emph{(Sines and Cosines)}
		For $z \in \mathbb{C}$, $\cos{(z)} = \frac{e^{i z} + e^{-i z} }{2}$, and $\sin{(z)} = \frac{e^{i z} - e^{-i z} }{2 i}$
	\end{definition}
	
	\begin{definition}
		A \textbf{curve} is a vector-valued function of a single variable; it is \textbf{smooth} if it is continuously differentiable. It is \textbf{piecewise smooth} if it is continuous and its domain, which is an interval, is a union of finite number of subintervals, and the curve is smooth on each of these.
	\end{definition}
	
	\begin{definition}
		A piecewise smooth simple closed curve is \textbf{positively oriented} if the inside of the curve is immediately to the left as you proceed around the curve in the given orientation.
	\end{definition}
	
	\begin{definition}
		If $\gamma : [a,b] \rightarrow \mathbb{C}$ is a continuous curve in $\mathbb{C}$, then the \textit{reversed curve}, denoted $-\gamma$, is another curve defined as:
		$$(-\gamma)(s) = \gamma(a + b - s).$$
	\end{definition}
	
	\begin{definition}
		\emph{(Concatenation of Curves)}
		If $\alpha : [a,b] \rightarrow \mathbb{C}$ and $\beta : [b,c] \rightarrow \mathbb{C}$ are curves with $\alpha(b) = \beta(b)$, the curve $\alpha + \beta : [a,c] \rightarrow \mathbb{C}$ is defined by:
		\begin{eqnarray*}
			\textbf (\alpha + \beta)(t) = \left\{\def\arraystretch{1.2}%
			\begin{array}{@{}c@{\quad}r@{}}
				\alpha(t) & \;\text{if $a \leq t \leq b$}\\
				\beta(t) & \;\text{if $b \leq t \leq c$}\\
			\end{array}\right.
		\end{eqnarray*}
	\end{definition}
	
	\begin{definition}
		A \textbf{VH chain} is a piecewise smooth curve whose pieces are alternately horizontal and vertical.
	\end{definition}
	
	\begin{definition}
		\emph{(Analytic Functions)}
		Suppose $D$ is a domain, $p \in D$, and $f : D \rightarrow \mathbb{C}$. Then $f$ is \textbf{differentiable} at $p$, with \textbf{derivative} $f^{\prime}(p) \in \mathbb{C}$ if:
		$$\lim_{z \rightarrow p} \frac{f(z) - f(p)}{z - p} = f^{\prime}(p) \; \; \bigg( = \lim_{h \rightarrow 0} \frac{f(z + h) - f(z)}{h} \bigg)$$
		and this limit exists. If $f$ is differentiable at every $p \in D$, then $f$ is said to be \textbf{analytic} on $D$. If $f$ is analytic on all of $\mathbb{C}$, then $f$ is called \textbf{entire}.
	\end{definition}
	
	\begin{definition}
		Suppose $U$ is an open subset of $\mathbb{C}$ and $w : U \rightarrow \mathbb{R}$. Then $w$ is \textbf{harmonic} on $U$ if $\forall z \in U, w_{xx}(z) + w_{yy}(z) = 0$.
	\end{definition}
	
	\begin{definition}
		If $D$ is a domain in $\mathbb{C}$, then two real-valued functions $u$ and $w$ on $D$ with continuous second order partials are \textbf{harmonic conjugates} if either is harmonic and they satisfy the Cauchy-Riemann equations.
	\end{definition}
	
	\begin{definition}
		A \textbf{power series} in $z \in \mathbb{C}$ is $\sum_{n=0}^{\infty} a_{n}(z-p)^{n}$, where $a_{n} \in \mathbb{C}$. The \textbf{radius of convergence} $0 \leq r < \infty$ of the power series is such that the series converges absolutely at any $z$ with $|z-p| < r$, and it diverges at $z$ if $|z-p| > r$.
	\end{definition}
	
	\begin{definition}
		A domain $D$ is called \textbf{simply connected} if $D$ contains the inside of every simple closed curve in $D$.
	\end{definition}
	
	\begin{definition}
		A \textbf{graph} $G$ in $\mathbb{C}$ is a finite collection of points, called the \textbf{vertices} of $G$ along with a finite collection of simple curves, called the \textbf{edges} of $G$, with the restrictions that both ends of any edge are among the vertices, and that any intersection of two edges occurs at an end of each.
	\end{definition}
	
	\begin{definition}
		A graph $G$ is a \textbf{directed graph} if each of its edges is oriented such that it has an \textbf{initial vertex} and a \textbf{terminal vertex}.
	\end{definition}
	
	\begin{definition}
		The \textbf{parity} of a vertex $v$ is the difference between the number of edges ending at $v$ and the number of edges beginning at $v$. A graph $G$ is called a \textbf{parity-0 graph} if the parity of every vertex in $G$ is 0.
	\end{definition}
	
	\begin{definition}
		An ordered collection of edges $(E_{i})$ of a graph $G$ is a \textbf{semisimple closed edge-path} if the $E_{j}$ are all distinct, the end of $E_{j}$ is the start of $E_{j+1}$, and the end of $E_{n}$ is the start of $E_{1}$. A semisimple closed edge-path is \textbf{simple} if the only times that the end of an edge and the start of an edge are the same are those listed above. Removing the edges of a semisimple closed edge path from a graph preserves the parity of the graph.
	\end{definition}
	
	\begin{definition}
		If $X$ is a nonempty subset of $\mathbb{C}$, the \textbf{diameter} of $X$ is the supremum of $|z-w|$ for all pairs of points $z, w$ of $X$ (this may be infinite).
	\end{definition}
	
	\begin{definition}
		Suppose $D$ is a domain and $f: D \rightarrow \mathbb{C}$ is analytic, but not constant. Let $p \in D$ and $f(p) = 0$. We say that $p$ is a \textbf{zero of order k} of $f$ if $k$ is the smallest natural number $n$ such that $f^{(n)}(p) \neq 0$.
	\end{definition}
	
	\begin{definition}
		We call $p$ a \textbf{singularity} of a function $f$ if $f$ is not differentiable at $p$. This can happen for two reasons:
		\begin{enumerate}[label=\alph*.]
			\item $f$ is not even defined at $p$
			\item $f$ is defined at $p$, but the limit defining the derivative does not exist.
		\end{enumerate}
	\end{definition}
	
	\begin{definition}
		$p$ is an \textbf{isolated singularity} of a function $f$ if $p$ is a singularity, yet there is some $r > 0$ such that $f$ is analytic on $B_{r}(p) \backslash \{p\}$ (a \textbf{"punctured disk"}). Here, there are three possibilities:
		\begin{enumerate}[label=\alph*.]
			\item $|f(z)| \rightarrow \infty$ as $z \rightarrow p$. In this case we say that $f$ has a \textbf{pole} at p.
			\item There is an $L > 0$ such that $|f(z)| < L$ for all $z \neq p$ near p. In this case we say that $f$ has a \textbf{removable singularity} at $p$.
			\item Neither (a) nor (b); in this case we say that $f$ has an \textbf{essential singularity} at $p$.
		\end{enumerate}
	\end{definition}
	
	\begin{definition}
		A series of the form $\sum a_{n} (z-p)^{n}$ where we allow both positive and negative values of $n$ is called a \textbf{Laurent series}.
	\end{definition}
	
	\begin{definition}
		Let $p$ be an isolated singularity of $f$, and let $\gamma$ be a small positively oriented circle centered at $p$. The \textbf{residue} of $f$ at $p$ is defined to be
		$$\text{Res}(f;p) = \frac{1}{2 \pi i} \int_{\gamma} f(z) dz.$$
	\end{definition}
	
	\begin{definition}
		A rational function $f(z) = \frac{P(z)}{Q(z)}$ is called a \textbf{simple rational function}, if the polynomials $P$ and $Q$ satisfy the following 3 conditions:
		\begin{enumerate}[label=\alph*.]
			\item degree of $P <$ degree of $Q$.
			\item $P$ and $Q$ have no roots in common.
			\item $Q$ has no repeated roots.
		\end{enumerate}
	\end{definition}
	
	%-------THEOREMS-------%
	
	\begin{theorem}
		\emph{(Properties of Addition and Multiplication)}
		Let $z$, $w$, $r$ $\in \mathbb{C}$. Then the following hold:
		\begin{itemize}
			\item $z + w = w + z$
			\item $zw = wz$
			\item $z + (w + r) = (z + w) + r$
			\item $z(wr) = (zw)r$
			\item $r(z + w) = rz + rw$
		\end{itemize}
	\end{theorem}
	
	\begin{theorem}
		\emph{(Properties of Conjugation)}
		Let $z$, $w$ $\in \mathbb{C}$. Let \normalfont{arg}($z$) = $\alpha$, and \normalfont{arg}($\overline{z}$) = $\alpha'$. Then the following hold:
		\begin{itemize}
			\item $\overline{\overline{z}} = z$
			\item $\alpha'$ = $-\alpha$
			\item $z \cdot \overline{z} = |z|^2 \geq 0$
			\item $\overline{z + w} = \overline{z} + \overline{w}$
			\item $\overline{zw} = \overline{z} \cdot \overline{w}$
		\end{itemize}
	\end{theorem}
	
	\begin{theorem}
		\emph{(Powers and Roots)}
		Let $z$ $\in \mathbb{C}$. Then the following hold:
		\begin{itemize}
			\item $|z^n| = |z|^n$
			\item \normalfont{arg}($z^n$) = $n \cdot$ \normalfont{arg($z$)}
		\end{itemize}
		If, additionally, $z$ is an $n^{th}$ root of $w$ $\in \mathbb{C}$, then
		\begin{itemize}
			\item $|z| = \sqrt[\leftroot{-2}\uproot{2}n]{|w|}$
			\item $n \cdot$ arg($z$) = arg($w$)
			\item $\alpha \in$ arg($w$) $\implies$ $\frac{\alpha}{n} + \frac{2 \pi k}{n} \in$ arg($z$) $\forall k \in \mathbb{Z}$
		\end{itemize}
	\end{theorem}
	
	\begin{corollary}
		Any nonzero complex number has exactly $n$ different $n^{th}$ roots.
	\end{corollary}
	
	\begin{theorem}
		\emph{(Polar Representation)}
		Let $z = x + iy \in \mathbb{C}$. Let $\alpha = $ arg($z$). The \textbf{polar representation} of $z$ is
		$$z = |z|(\cos{\alpha} + i \sin{\alpha}).$$
	\end{theorem}
	
	\begin{theorem}
		\emph{(De Moivre)}
		For all $t \in \mathbb{R}$, $n \in \mathbb{N}$,
		$$[\cos{(t)} + i \sin{(t)}]^{n} = \cos{(nt)} + i \sin{(nt)}.$$
	\end{theorem}
	
	\begin{theorem}
		Let $z$, $w$ $\in \mathbb{C}$. $z$ is perpendicular to $w$ if and only if $\Re(\overline{z}w) = 0$.
	\end{theorem}
	
	\begin{theorem}
		Let $p \in \mathbb{C}$, $a \in \mathbb{R}$, $a > 0$, $u = \cos{\theta} + i \sin{\theta}$. Then, $T_{p}$, $R_{\theta}$, $S_{a}$, and $M_{p}$ preserve angles, map circles to circles and lines to lines, and are invertible. Their inverses are, respectively, $T_{-p}$, $R_{-\theta}$, $S_{\frac{1}{a}}$, and $M_{\frac{1}{p}}$. 
	\end{theorem}
	
	\begin{corollary}
		Let $b \in \mathbb{C}$, $a = |b|$, $\theta \in \text{arg}(b)$. Then, $M_{b} = S_{a} \circ R_{\theta}$.
	\end{corollary}
	
	\begin{theorem}
		The inversion mapping $J$ is invertible, and is its own inverse. However, it need not map circles to circles, or lines to lines.
	\end{theorem}
	
	\begin{lemma}
		\emph{(Inversion)}
		The following holds $\forall z \in \mathbb{C}$:
		\begin{itemize}
			\item $J(M_{W}(z)) = M_{\frac{1}{w}}(J(z))$
			\item $J(R_{\theta}(z)) = R_{-\theta}(J(z))$
			\item $J(S_{a}(z)) = S_{\frac{1}{a}}(J(z))$
			\item $\overline{J(z)} = J(\overline{z})$
		\end{itemize}
	\end{lemma}
	
	\begin{theorem}
		The inversion mapping $J$ takes any circle not passing through zero to another circle not passing through zero.
	\end{theorem}
	
	\begin{theorem}
		\emph{(More About Circles)}
		If $b \neq 0$ is in $\mathbb{C}$ and $r > 0$, but $r \neq 1$, then:
		\begin{enumerate}[label=\alph*.]
			\item The set $\{z \in \mathbb{C} \; \, | \; \,  r|z| = |z - b|\}$ is a circle.
			\item The center of the circle is $\frac{b}{1 - r^2}$ and its radius is $\frac{r|b|}{|1-r^2|}$.
			\item 0 and b are on opposite sides of this circle.
		\end{enumerate}
	\end{theorem}
	
	\begin{corollary}
		\label{circles}
		If $p \neq q$ and $0 < r < 1$, then $r|z - p| = |z - q|$ is a circle. Additionally:
		\begin{enumerate}[label=\alph*.]
			\item Its center is $\frac{q - pr^2}{1 - r^2}$.
			\item Its radius is $\frac{r|q - p|}{1 - r^2}$.
			\item $p$ and $q$ are on opposite sides of the circle.
		\end{enumerate}
	\end{corollary}
	
	\begin{corollary}
		In Corollary \ref{circles}, if $q > 0$ and $p = -q$, then:
		\begin{enumerate}[label=\alph*.]
			\item The center is at $v = q \frac{1+r^2}{1-r^2} > q > 0$.
			\item The radius is $\frac{2r|q|}{1-r^2}$.
			\item $q$ is inside the circle, and $-q$ is outside.
		\end{enumerate}
	\end{corollary}
	
	\begin{theorem}
		\emph{(Open and Closed Sets)}
		\begin{enumerate}[label=\alph*.]
			\item $S$ and $\mathbb{C} \setminus S$ have the same boundary.
			\item $S$ is open $\iff$ $S \cap \partial S = \varnothing$.
			\item $S$ is closed $\iff$ $\partial S \subset S$.
		\end{enumerate}
	\end{theorem}
	
	\begin{theorem}
		\emph{(Compactness)}
		Subsets of $\mathbb{R}^n$ or $\mathbb{C}$ are compact if and only if they are closed and bounded.
	\end{theorem}
	
	\begin{theorem}
		$f(z) \rightarrow L \iff \Re{[f(z)]} \rightarrow \Re{[L]}$ and $\Im{[f(z)]} \rightarrow \Im{[L]}$.
	\end{theorem}
	
	\begin{theorem}
		\emph{(Limits and Arithmetic)}
		Suppose that as $z \rightarrow p$ (or $z \rightarrow \infty$), both $f(z) \rightarrow L$ and $g(z) \rightarrow M$. Then:
		\begin{enumerate}[label=\alph*.]
			\item $f(z) + g(z) \rightarrow L + M$
			\item $f(z)g(z) \rightarrow LM$
			\item As long as $M \neq 0$, $\frac{f(z)}{g(z)} \rightarrow \frac{L}{M}.$
		\end{enumerate}
	\end{theorem}
	
	\begin{theorem}
		A function $f$ is continuous at $p$ if and only if $\forall \epsilon > 0$, $\exists \delta > 0$ such that $|z - p| < \delta \implies |f(z) - f(p)| < \epsilon$.
	\end{theorem}
	
	\begin{theorem}
		Under the appropriate assumptions, continuous functions preserve continuity under addition, multiplication, division, and composition.
	\end{theorem}
	
	\begin{theorem}
		$f: \mathbb{C} \rightarrow \mathbb{C}$ is continuous if and only if $\forall$ open $U \subset \mathbb{C}$, $f^{-1}(U)$ is also open.
	\end{theorem}
	
	\begin{theorem}
		Suppose $X$ is a compact subset of $\mathbb{C}$ and $f : X \rightarrow \mathbb{C}$ is continuous. Then $f(X)$ is compact.
	\end{theorem}
	
	\begin{corollary}
		If $X \neq \varnothing$ is compact and $f : X \rightarrow \mathbb{R}$ is continuous, then $f$ achieves its maximum and minimum values on $X$.
	\end{corollary}
	
	\begin{theorem}
		\emph{(Convergence Tests)}
		\begin{enumerate}[label=\alph*.]
			\item $\sum_{n=1}^{\infty} a_{n} \rightarrow L \implies a_{n} \rightarrow 0$.
			\item $a_{n} \nrightarrow 0 \implies \sum a_{n}$ does not converge.
			\item A series of nonnegative terms either converges or diverges to $\pm \infty$
			\item For real or complex numbers $x_{n}$, if $\sum |x_{n}|$ converges, so does $\sum x_{n}$.
			\item If $|r| < 1$ then $\sum_{n=o}^{\infty} r^{n}$ converges to $\frac{1}{1 - r}$. Otherwise, the series diverges.
		\end{enumerate}
	\end{theorem}
	
	\begin{theorem}
		\emph{(Properties of the Complex Exponential)}
		\begin{enumerate}[label=\alph*.]
			\item $|e^{z}| = e^{x}$
			\item $\text{arg}(e^{z}) = y$
			\item $e^{0} = 1$
			\item $e^{z}e^{w} = e^{z+w}$
			\item $e^{-z}$ = $\frac{1}{e^{z}}$
			\item $e^{z} = e^{z + 2 \pi i}$
			\item $e^{\overline{z}} = \overline{e^{z}}$
			\item $e^{i \pi} = -1$
		\end{enumerate}
	\end{theorem}
	
	\begin{theorem}
		$\log(zw) = \log(z) + \log(w)$, as sets.
	\end{theorem}
	
	\begin{theorem}
		\emph{(Properties of Hyperbolic Functions)}
		For $z \in \mathbb{C}$:
		\begin{enumerate}[label=\alph*.]
			\item $\cosh{(-z)} = \cosh{(z)}$
			\item $\sinh{(-z)} = -\sinh{(z)}$
			\item $\cosh^{2}{(z)} - \sinh^{2}{(z)} = 1$
			\item For $t \in \mathbb{R}$, $[\cosh{(t)}]' = \sinh{(t)}$ and $[\sinh{(t)}]' = \cosh{(t)}$
		\end{enumerate}
	\end{theorem}
	
	\begin{theorem}
		\emph{(Properties of Sines and Cosines)}
		For $z \in \mathbb{C}$:
		\begin{enumerate}[label=\alph*.]
			\item $\sin^{2}{(z)} + \cos^{2}{(z)} = 1$
			\item $\sin{(-z)} = -\sin{(z)}$
			\item $\cos{(-z)} = \cos{(z)}$
			\item $\cos{(\overline{z})} = \overline{\cos{(z)}}$, $\sin{(\overline{z})} = \overline{\sin{(z)}}$
			\item $\cos{(z)} = \cosh{(i z)}$
			\item $\sin{(z)} = -i\sinh{(i z)}$
			\item $\cos{(i z)} = \cosh{(z)}$
			\item $\sin{(i z)} = i \sinh{(z)}$
			\item $\cos$, $\sin$ are unbounded.
		\end{enumerate}
	\end{theorem}
	
	\begin{theorem}
		\emph{(Trig Identities)}
		For $z, w \in \mathbb{C}$:
		\begin{enumerate}[label=\alph*.]
			\item $\cos{(z)} + i \sin{(z)} = e^{i z}$
			\item $\sin{(z)}\cos{(w)} + \sin{(w)}\cos{(z)} = \sin{(z + w)}$
			\item $\sin{(w + \frac{\pi}{2})} = \cos{(w)}$
			\item $\sin{(w + \pi)} = -\sin{(w)}$
			\item $\cos{(z + w)} = \cos{(z)}\cos{(w)} - \sin{(z)}\sin{(w)}$
		\end{enumerate}
	\end{theorem}
	
	\begin{theorem}
		\emph{(Jordan Curve Theorem)}
		The complement of the image of a simple closed curve in $\mathbb{C}$ is a pair of disjoint domains, and the image of the curve is the boundary of each of these. Of the two domains, one is bounded (the
		inside of the curve), and the other is unbounded (the outside of the curve).
	\end{theorem}
	
	\begin{theorem}
		Let $z = x + iy \in \mathbb{C}$. The principal argument of $z$ is:
		\begin{eqnarray*}
			\textbf{Arg}(z)= \left\{\def\arraystretch{1.2}%
			\begin{array}{@{}c@{\quad}r@{}}
				\arctan (\frac{y}{x}) & \;\text{if $x>0$, $y\in \mathbb R$}\\
				\arctan (\frac{y}{x})+\pi  & \;\text{if $x <0$, $y\geq 0$}\\
				\arctan (\frac{y}{x})-\pi  & \;\text{if $x < 0$, $y < 0$}\\
				\frac{\pi}{2} & \;\text{if $x=0$, $y> 0$}\\
				-\frac{\pi}{2} & \;\text{if $x=0$, $y < 0$}\\
				\text{undefined}& \;\text{if $x=0$, $y=0$}\\
			\end{array}\right.
		\end{eqnarray*}
	\end{theorem}
	
	\begin{theorem}
		\emph{(Integral along a parameterized curve)}
		If $\gamma : [a,b] \rightarrow \mathbb{C}$ is continuous, then 
		$$\int_{a}^{b} \gamma (t) \, dt = \int_{a}^{b} \Re{[\gamma(t)] \, dt} + i \int_{a}^{b} \Im{[\gamma(t)] \, dt}.$$
	\end{theorem}
	
	\begin{theorem}
		\emph{(Integral of a function along a parameterized curve)}
		If $\gamma : [a,b] \rightarrow \mathbb{C}$ is a smooth curve with image $S \subset \mathbb{C}$, and $f : S \subset \mathbb{C} \rightarrow \mathbb{C}$ is a continuous function, then the line integral of $f$ over $\gamma$ is 
		$$\int_{a}^{b} f(\gamma (t)) \gamma^{\prime}(t) \, dt.$$
	\end{theorem}
	
	\begin{theorem}
		\emph{(Linearity Properties of Line Integrals)}
		\begin{enumerate}[label=\alph*.]
			\item $\int_{\gamma} f + g \, dz = \int_{\gamma} f \, dz + \int_{\gamma} g \, dz$
			\item $\forall c \in \mathbb{R}, \int_{\gamma} c f \, dz = c \int_{\gamma} f \, dz$
		\end{enumerate}
	\end{theorem}
	
	\begin{theorem}
		\emph{(Line Integral of Reversed Curve)}
		$$\int_{-\gamma} f \, dz = - \int_{\gamma} f \, dz$$
	\end{theorem}
	
	\begin{theorem}
		\emph{(Line Integral over Concatenated Curves)}
		$$\int_{\alpha + \beta} f = \int_{\alpha} f + \int_{\beta} f$$
	\end{theorem}
	
	\begin{theorem}
		Suppose $\phi : [a,b] \rightarrow \mathbb{C}$ is continuous. Then:
		$$\int_{a}^{b}|\phi(t)| \, dt \geq \bigg | \int_{a}^{b}\phi(t) \, dt \bigg |$$
	\end{theorem}
	
	\begin{corollary}
		If $f$ is continuous and $\gamma$ is piecewise smooth, then
		$$\bigg | \int_{\gamma} f \, dz \bigg | \leq (\max | f(\gamma(t)) |) (\text{length of} \gamma).$$
	\end{corollary}
	
	\begin{lemma}
		\emph{(Product Rule for Differentiable Curves)}
		If $f$, $g : [a,b] \rightarrow \mathbb{C}$ are both differentiable curves, then so is $fg$, and
		$$(fg)^{\prime} = f^{\prime}g + fg^{\prime}.$$
	\end{lemma}
	
	\begin{corollary}
		Suppose $g : [a,b] \rightarrow \mathbb{C}$ is a differentiable curve. Then:
		\begin{enumerate}[label=\alph*.]
			\item So is $g \overline{g} = |g|^{2}.$
			\item So is $g^{m}$ for any $m \geq 1$, and $[g^{m}]^{\prime} = m g^{m-1} g^{\prime}.$
		\end{enumerate}
	\end{corollary}
	
	\begin{theorem}
		\emph{(Quotient Rule for Curves)}
		If $f: [a,b] \rightarrow \mathbb{C}$ is a differentiable curve that is never 0, then $\frac{1}{f}$ is also a differentiable curve, and
		$$\bigg(\frac{1}{f}\bigg)^{\prime} =  \frac{-f^{\prime}}{f^2}.$$
	\end{theorem}
	
	\begin{corollary}
		Suppose $f,g: [a,b] \rightarrow \mathbb{C}$ are differentiable curves and $g$ is never zero.
		\begin{enumerate}[label=\alph*.]
			\item $\frac{f}{g}$ is a differentiable curve, and $\big(\frac{f}{g}\big)^{\prime} = \frac{f^{\prime}g - f g^{\prime}}{g^2}$
			\item For $m \geq 1$, $\frac{1}{g^m}$ is a differentiable curve, and $(g^{-m})^{\prime} = -mg^{-m-1}g^{\prime}$
		\end{enumerate}
	\end{corollary}
	
	\begin{theorem}
		\emph{(Line integral of $z^{m}$ over a curve)}
		Suppose $\gamma : [a,b] \rightarrow \mathbb{C}$ is piecewise smooth and $m \neq -1$ is an integer. Then,
		$$\int_{\gamma} z^{m} dz = \frac{\gamma(b)^{m+1} - \gamma(a)^{m+1}}{m+1}.$$
	\end{theorem}
	
	\begin{corollary}
		Suppose $\gamma$ is a piecewise smooth closed curve, and $m \neq -1$ is an integer. Then,
		$$\int_{\gamma} z^{m} dz = 0.$$
	\end{corollary}
	
	\begin{theorem}
		\emph{(Green's Theorem)}
		Let $\{\gamma_{j}\}$ be a finite disjoint collection of piecewise smooth simple closed curves. Let each $\gamma_{j}$ be inside of some curve $\gamma_{0}$, and any two of the $\gamma_{j}$'s outside of each other. Let $D$ denote the region inside of $\gamma_{0}$ and outside all of the other $\gamma_{j}$'s. Define $\partial D = \cup_{k=0}^{n} \gamma_{k}$ with a positive orientation. Define $\int_{\partial D} f dz$ to mean $\sum_{k=0}^{n} \int_{\gamma_{k}} f dz$. Let $G$ be an open set containing $D$ and $\partial D$. Let $f : G \rightarrow \mathbb{C}$ be continuous and write $f(z) = f(x + iy) = u(x + iy)  + iw(x + iy)$. Suppose the partial derivatives $u_{x}, u_{y}, w_{x}, w_{y}$ exist and are continuous on $G$. Then,
		$$\int_{\partial D} f dz = i \int \int_{D} \bigg[\frac{\partial f}{\partial x} + i \frac{\partial f}{\partial y}\bigg] dxdy = i\bigg(\int \int_{D} \frac{\partial f}{\partial x} dxdy\bigg) - \int \int_{D} \frac{\partial f}{\partial y} dxdy.$$
	\end{theorem}
	
	\begin{theorem}
		Suppose $\gamma$ is a piecewise smooth, positively oriented, simple closed curve, and $p \in \mathbb{C}$ is not in the image of $\gamma$. Then:
		\begin{eqnarray*}
			\int_{\gamma} \frac{1}{z - p} dz = \left\{\def\arraystretch{1.2}%
			\begin{array}{@{}c@{\quad}r@{}}
				2 \pi i & \;\text{if $p \in \gamma$}\\
				0 & \;\text{otherwise}\\
			\end{array}\right.
		\end{eqnarray*}
	\end{theorem}
	
	\begin{theorem}
		If $D$ is an open disk containing $p$ and $q$, there there is a VH chain in $D$ from $p$ to $q$.
	\end{theorem}
	
	\begin{theorem}
		Suppose $p, q \in \mathbb{C}$ and that $D$ is an open set containing the closed line segment $L$ going from $p$ to $q$. Then there is a VH chain in $D$ going from $p$ to $q$.
	\end{theorem}
	
	\begin{theorem}
		\emph{(Characterization of domains)}
		An open set $D$ is a domain if and only if $\forall \, p, q \in D, \exists$ a VH chain in $D$ going from $p$ to $q$.
	\end{theorem}
	
	\begin{theorem}
		\emph{(Characterization of constant functions on domains)}
		Suppose $D$ is a domain and $u : D \rightarrow \mathbb{R}$ is smooth. If both $u_{x}, u_{y}$ are identically 0 on $D$, then $u$ is constant on $D$.
	\end{theorem}
	
	\begin{theorem}
		\emph{(Sums and constant multiples of differentiable functions)}
		Suppose that $f^{\prime}(p)$ and $g^{\prime}(p)$ exist, and that $c \in \mathbb{C}$ is a constant. Then:
		\begin{enumerate}[label=\alph*.]
			\item $f + g$ is differentiable at $p$, with derivative $f^{\prime}(p) + g^{\prime}(p)$
			\item $c f$ is differentiable at $p$, with derivative $c \cdot f^{\prime}(p)$
		\end{enumerate}
	\end{theorem}
	
	\begin{theorem}
		\emph{(Continuity Estimate)}
		Suppose $f$ is differentiable at $p$. Let $\epsilon > 0$. Then there is a $\delta > 0$ such that for any $z$ with $|z - p| < \delta$,
		$$|f(z) - f(p)| \leq (|f^{\prime}(p)| + \epsilon) |z - p|.$$
		It follows that if $f$ is differentiable at $p$, then $f$ is continuous at $p$.
	\end{theorem}
	
	\begin{theorem}
		\emph{(Product Rule)}
		If $f(z)$ and $g(z)$ are both differentiable at $p$, then so is $h(z) = f(z)g(z)$, and
		$$h^{\prime}(p) = f^{\prime}(p)g(p) + f(p)g^{\prime}(p).$$
	\end{theorem}
	
	\begin{theorem}
		\emph{(Chain Rule)}
		If $f$ is differentiable at $p$ and $g$ is differentiable at $p$, then $g \circ f$ is differentiable at $p$, and
		$$(g \circ f)^{\prime}(p) = g^{\prime}(f(p))f^{\prime}(p).$$
	\end{theorem}
	
	\begin{theorem}
		\emph{(Quotient Rule)}
		Suppose $f$ and $g$ are functions that are both differentiable at $p$, and that $g(p) \neq 0$. Let $h(z) = \frac{f(z)}{g(z)}$. Then $h$ is differentiable at $p$, and
		$$h^{\prime}(p) = \frac{f^{\prime}(p)g(p) - f(p)g^{\prime}(p)}{g(p)^2}.$$
	\end{theorem}
	
	\begin{theorem}
		\emph{(Cauchy-Riemann Equations)}
		Suppose $D \subset \mathbb{C}$ is a domain, and that $f : D \rightarrow \mathbb{C}$ is analytic. Write
		$$f(z) = f(x + iy) = u(x + iy) + iw(x + iy)).$$
		Then the partial derivatives of $u$ and $w$ with respect to $x$ and $y$ exist, $u_{x} = w_{y}$, and $u_{y} = -w_{x}$, so that $u_{x}u_{y} + w_{x}w_{y} = 0$.
	\end{theorem}
	
	\begin{theorem}
		Suppose that $u, w$ are real-valued functions defined on a domain $D$, that the second order partial derivatives of $u$ and $w$ exist and are continuous, and that $u, w$ satisfy the Cauchy-Riemann equations. Then $u$ and $w$ are harmonic on $D$.
	\end{theorem}
	
	\begin{corollary}
		If $f = u + iw$ is analytic on a domain $D$, then $u$ and $w$ are harmonic.
	\end{corollary}
	
	\begin{theorem}
		\emph{(Cauchy-Riemann implies analyticity)}
		Suppose $D$ is an open disk centered at $p$, and that $f = u + iw$ is a function from $D$ to $\mathbb{C}$. If the Cauchy-Riemann equations for $u$ and $w$ hold at $p$, then $f$ is analytic at $p$, and $f^{\prime}(p) = u_{x}(p) + iw_{x}(p)$.
	\end{theorem}
	
	\begin{theorem}
		\emph{(Harmonic conjugates always exist on disks)}
		Suppose $D \subset \mathbb{C}$ is a disk, and that $u : D \rightarrow \mathbb{R}$ is harmonic. Then there is another function $w : D \rightarrow \mathbb{R}$ such that the function $f(z) = u(z) + iw(z)$ satisfies the Cauchy-Riemann equations.
	\end{theorem}
	
	\begin{theorem}
		\emph{(Restrictions on Analytic Functions)}
		Suppose $f = u + iw$ is an analytic function on a domain $D$.
		\begin{enumerate}[label=\alph*.]
			\item If $u$ is constant on $D$, then so is $f$.
			\item If $u^2 + w^2$ is constant on $D$, then so is $f$.
		\end{enumerate}
	\end{theorem}
	
	\begin{theorem}
		If $\sum_{n=0}^{\infty} a_{n}(z - p)^{n}$ converges at $z$, then it converges absolutely at $w$ whenever $|w - p| < |z - p|$.
	\end{theorem}
	
	\begin{corollary}
		If a power series $\sum_{n=0}^{\infty} a_{n}(z - p)^{n}$ diverges at $z$, then it also diverges at any $w$ with $|w-p| > |z-p|$.
	\end{corollary}
	
	\begin{theorem}
		\emph{(Ratio and Root Tests)}
		Consider the power series $$\sum_{n=0}^{\infty} a_{n}(z - p)^{n}$$.
		\begin{enumerate}[label=\alph*.]
			\item Suppose $\lim_{n \rightarrow \infty} \sqrt[n]{|a_{n}|}$ exists; call it $L$. If $L > 0$ then the radius of convergence is $\frac{1}{L}$; if $L = 0$ then the radius of convergence is infinite; if the limit is infinite, then the radius of convergence is 0.
			\item Suppose $\lim_{n \rightarrow \infty} \frac{|a_{n+1}|}{|a_{n}|}$ exists; call it $L$. Then the radius of convergence is as above.
		\end{enumerate}
	\end{theorem}
	
	\begin{lemma}
		Suppose that the radius of convergence of $\sum a_{n}(z - p)^{n}$ is $R > 0$. Then the radius of convergence of $\sum n a_{n} (z-p)^{n-1}$ is at least $R$.
	\end{lemma}
	
	\begin{theorem}
		\emph{(Derivatives of power series)}
		If a power series $\sum_{n=0}^{\infty} a_{n}(z - p)^{n}$ has radius of convergence $R > 0$ then we can define a function $f$ on the open disk $D$ of radius $R$ centered at $p$ by $f(z) = \sum_{n=0}^{\infty} a_{n}(z - p)^{n}$. Then:
		\begin{enumerate}[label=\alph*.]
			\item $f$ is analytic on $D$
			\item $\forall z \in D, f^{\prime}(z)$ is given by the power series $\sum_{n=0}^{\infty} n a_{n}(z - p)^{n-1}$
		\end{enumerate}
	\end{theorem}
	
	\begin{corollary}
		\emph{(Derivatives of power series)}
		\begin{enumerate}[label=\alph*.]
			\item Every derivative of $f$ exists and is given as a power series with radius of convergence at least $R$.
			\item The power series for the derivatives can be found by termwise differentiation.
			\item $$a_{n} = \frac{f^{n}(p)}{n!}$$
			\item A power series with a positive radius of convergence is just a Taylor series:
			$$\sum_{n=0}^{\infty} a_{n}(z - p)^{n} = \sum_{n=0}^{\infty} \frac{f^{n}(p)(z-p)^{n}}{n!}$$
		\end{enumerate}
	\end{corollary}
	
	\begin{lemma}
		Suppose that $G$ is a parity-0 graph with at least one edge. Then $G$ contains a simple closed edge-path $L$ with at least one edge.
	\end{lemma}
	
	\begin{theorem}
		Suppose that $G$ is a parity-0 graph whose set of edges $E$ is nonempty. The $E$ is the disjoint union of simple closed edge-paths $\{L_{j}\}$.
	\end{theorem}
	
	\begin{lemma}
		If $D$ is a domain containing the closed line segment $K$ joining $p$ and $q$, there there is a positive constant $\epsilon > 0$ such that for any point $z \in K$, the ball of radius $\epsilon$ about $z$ is contained in $D$.
	\end{lemma}
	
	\begin{theorem}
		If $D$ is a simply connected domain and $f : D \rightarrow \mathbb{C}$ is analytic, then there is an analytic $g : D \rightarrow \mathbb{C}$ with $g^{\prime} = f$.
	\end{theorem}
	
	\begin{lemma}
		\emph{(Green analyticity lemma)}
		Suppose that $D$ is a domain and that $f : D \rightarrow \mathbb{C}$ is analytic. Then at any point of $D$,
		$$f_{x} + if_{y} = 0.$$
	\end{lemma}
	
	\begin{theorem}
		\emph{(Cauchy's Formula)}
		Suppose $D \subset \mathbb{C}$ is a domain, that $f : D \rightarrow \mathbb{C}$ is analytic, and that $\gamma$ is a positively-oriented, piecewise smooth, simply connected curve in $D$, such that the inside of $\gamma$ is also in $D$. If $p$ is a point inside of $\gamma$, then:
		$$\frac{1}{2\pi i} \int_{\gamma} \frac{f(z)}{z - p} dz = f(p).$$
	\end{theorem}
	
	\begin{theorem}
		For a triangle $\mathcal{T} = PQR$ (including its inside), the diameter is the length of its longest side.
	\end{theorem}
	
	\begin{theorem}
		Suppose $K_{n} \subset \mathbb{C}$ are closed nonempty sets, with $K_{n+1} \subset K_{n}$ for each $n$, and that the diameter of $K_{n}$ goes to $0$ as $n \rightarrow \infty$. Then, $\cup K_{n}$ is a single point.
	\end{theorem}
	
	\begin{theorem}
		\emph{(Cauchy-Goursat Theorem)}
		Suppose $D \subset \mathbb{C}$ is a domain, $f : D \rightarrow \mathbb{C}$ is analytic, and that $T$ is a triangular simple closed curve in $D$ whose inside is also in $D$. Then,
		$$\int_{T}f dz = 0.$$
	\end{theorem}
	
	\begin{theorem}
		\emph{(General Cauchy Theorem)}
		Suppose $D \subset \mathbb{C}$ is a simply connected domain, and that $f : D \rightarrow \mathbb{C}$ is analytic. Then,
		\begin{enumerate}[label=\alph*.]
			\item There is an analytic function $g : D \rightarrow \mathbb{C}$ such that $g^{\prime} = f$.
			\item If $\gamma$ is a piecewise smooth closed curve in $D$, then $\int_{\gamma}f dz = 0$.
		\end{enumerate}
	\end{theorem}
	
	\begin{theorem}
		\emph{(Extending Cauchy's Theorem)}
		Suppose $U \subset \mathbb{C}$ is open, $f : U \rightarrow \mathbb{C}$, and $f^{\prime}$ is continuous on $U$. For $p \in U$, choose $r > 0$ so that $B_{r}(p) \subset U$.
		\begin{enumerate}[label=\alph*.]
			\item There is a power series $\sum a_{n} (z-p)^{n}$ with radius of convergence at least $r$ such that $f(z) = \sum a_{n} (z-p)^{n}$ for every $z$ with $|z-p| < r$.
			\item The coefficients $a_{n}$ are given by
			$$a_{n} = \frac{1}{2 \pi i} \int_{\gamma} \frac{f(w)}{(w-p)^{n+1}} dw$$
			where $\gamma$ is any counterclockwise circle centered at $p$ with radius $d < r$.
		\end{enumerate}
	\end{theorem}
	
	\begin{theorem}
		\emph{(General Cauchy Formula)}
		Suppose $p \in \mathbb{C}$, $r > 0$, and $f : U = B_{r}(p) \rightarrow \mathbb{C}$ analytic. Let $\gamma$ be a counterclockwise circle centered at p with radius $d < r$. Define
		$$a_{n} = \frac{1}{2 \pi i} \int_{\gamma} \frac{f(w)}{(w-p)^{n+1}} dw.$$
		Then $f(z) = \sum_{n=0}^{\infty} a_{n} (z-p)^{n}$ for every $z$ with $|z-p| < r$, and the radius of convergence of this power series is at least $r$.
	\end{theorem}
	
	\begin{theorem}
		\emph{(Liouville)}
		If $f : \mathbb{C} \rightarrow \mathbb{C}$ is bounded and entire, then $f$ is constant.
	\end{theorem}
	
	\begin{theorem}
		\emph{(Morera)}
		Suppose $D$ is a domain, that $f : D \rightarrow \mathbb{C}$ is continuous, and that if $T$ is any triangle in $D$ whose inside is also in $D$, then the line integral of $f$ around $T$ is 0. Then $f$ is analytic on $D$.
	\end{theorem}
	
	\begin{theorem}
		\emph{(Removable singularities are removable)}
		If f has a removable singularity at $p$, then it is possible to make $f$ become analytic at $p$ by redefining $f(p)$.
	\end{theorem}
	
	\begin{theorem}
		\emph{(Laurent series near poles)}
		If f has a pole at $p$, then for any large positive integer $m$, $(z-p)^{m} f(z)$ has a removable singularity at $p$. The smallest such $m$ is called the \textbf{order} of the pole.
	\end{theorem}
	
	\begin{theorem}
		If $p$ is an essential singularity of $f$, then there exists a Laurent series for $f$
		in powers of $(z-p)$, and it must contain arbitrarily high powers of $\frac{1}{(z-p)}$.
	\end{theorem}
	
	\begin{theorem}
		\emph{(Residues)}
		Suppose $p$ is an isolated singularity of $f$. Then for all sufficiently small $r > 0$, the line integral of $f$ over the positively oriented circle of radius $r$ centered at $p$ is independent of $r$. 
	\end{theorem}
	
	\begin{lemma}
		\emph{(Residues at removable singularities)}
		If $f$ has a removable singularity at $p$, then $\text{Res}(f;p) = 0$.
	\end{lemma}
	
	\begin{lemma}
		\emph{(Residues at poles)}
		Suppose that $p$ is a pole of order $m$ for $f$. It is known that, for $z$ near $p$, $f(z)$ is given by a Laurent series
		$$f(z) = \sum_{j=-m}^{\infty} a_{j} (z-p)^{j}$$.
		Then $\text{Res(f;p)} = a_{-1}$.
	\end{lemma}
	
	\begin{corollary}
		If $p$ is a pole of order $m$ for $f$, and $g$ is the function that agrees with $(z-p)^{m} f(z)$ near $p$ and is analytic on a disc about $p$, then the Laurent coefficient $a_{-1}$ for $f$ is the coefficient of $(z-p)^{m-1}$ in the Taylor series for $g$:
		$$a_{-1} = \frac{g^{(m-1)}(p)}{(m-1)!}$$.
	\end{corollary}
	
	\begin{theorem}
		\emph{(Residue Theorem)}
		Suppose $D \subset \mathbb{C}$ is a simply connected domain and $f : D \rightarrow \mathbb{C}$ is analytic except for finitely many isolated singularities $p_{j}$. Suppose $\gamma$ is a positively oriented, piecewise smooth, simple closed curve in $D$ that does not hit any of the $p_{j}$. Let $U$ be the inside of $\gamma$. Then
		$$\int_{\gamma}f(z) dz = 2 \pi i \sum_{p_{j} \in U}\text{Res}(f;p_{j}).$$
	\end{theorem}
	
	\begin{theorem}
		\emph{(Laurent series on annuli)}
		Suppose $0 \leq r < R$ and that $f$ is analytic on the open annulus $A$ given by $r < |z| < R$. Then there are two analytic functions given by series:
		\begin{enumerate}[label=\alph*.]
			\item $f_{1}(z) = \sum_{n=0}^{\infty} a_{n} z^{n}$, converging on $|z| < R$
			\item $f_{2}(z) = \sum_{n=1}^{\infty} \frac{b_{n}}{z_{n}}$
		\end{enumerate}
		such that $f(z) = f_{1}(z) + f_{2}(z)$ for each $z \in A$.
	\end{theorem}
	
	\begin{corollary}
		For any $b$ with $r < b < R$, if for each integer $n$ we set
		$$a_{n} = \frac{1}{2 \pi i} \int_{|w| = b} \frac{f(w)}{w^{n+1}} dw$$,
		then $f_{1}(z) = \sum_{n \geq 0} a_{n} z^{n}$ and $f_{2}(z) = \sum_{n \leq -1} a_{n} z^{n}$.
	\end{corollary}
	
	\begin{lemma}
		Suppose $D$ is a domain and $f: D \rightarrow \mathbb{C}$ is analytic except for finitely many poles $p_{j}$, of order $m_{j}$. Then there is an analytic function $g(z)$ on $D$ and for each $p_{j}$ a polynomial $h_{j}$ of degree $m_{j}$, with no constant term, such that on $D \ \{ p_{j} \}$,
		$$f(z) = g(z) + \sum_{j} h_{j} \frac{1}{(z-p_{j})}$$
	\end{lemma}
	
	\begin{theorem}
		\emph{(Partial fractions for simple rational functions)}
		Suppose $f(z) = \frac{P(z)}{Q(z)}$ is a simple rational function. Let $c_{n}$ be the $n^{th}$ root of $Q$. Then
		$$f(z) = \sum_{n} \frac{\text{Res}(f; c_{n})}{z - c_{n}}.$$
	\end{theorem}
	
	\begin{theorem}
		\emph{(Real integrals using residues)}
		Suppose $P$ and $Q$ are \textbf{real} polynomials, with deg($Q$) $\geq$ $2 + \text{deg}(P)$, and that $Q$ has no real roots. Let $f(x) = \frac{P(x)}{Q(x)}$. Let $U$ be the set of roots of $Q$ that are in the upper half plane in $C$. Then
		$$\int_{-\infty}^{\infty} f(x) dx = 2 \pi i \sum_{q \in U} \text{Res}(f; q).$$
	\end{theorem}
	
	\begin{lemma}
		\emph{(Jordan)}
		Let $\gamma(t) = r e^{it}$ for $0 \leq t \leq \pi$. If $g(z)$ is continuous and $|g(z)| \leq B_{r}$ for all $z$ on $\gamma$, then $|\int_{\gamma} e^{iz} g(z) dz| \leq \pi B_{r}$.
	\end{lemma}
	
	\begin{lemma}
		Suppose $D \subset \mathbb{C}$ is a domain containing a point $p$, that $g$ is an analytic function on $D$ with $g(p) \neq 0$, and that $f(z) = (z-p)^{k}g(z)$ for some integer $k \neq 0$.
		\begin{enumerate}[label=\alph*.]
			\item If $k > 0$ then $f$ has a zero of order $k$ at $p$.
			\item If $k < 0$ then $f$ has a pole of order $|k| = -k$ at $p$.
			\item In either case, $p$ is a pole of $\frac{f^{\prime}}{f}$ of order $1$.
			\item $\text{Res}(\frac{f^{\prime}}{f}; p) = k$.
		\end{enumerate}
	\end{lemma}
	
	\begin{theorem}
		Suppose $D$ is a domain, $f : D \rightarrow \mathbb{C}$ is non-constant and analytic except at a finite set $P \subset D$ of poles. Let $Z$ be the set of $zeros$ of $f$ in $D$. Assume $\gamma : [a,b] \rightarrow D$ is a piecewise smooth, positively oriented, simple closed curve in $D$ that does not meet $P$ or $Z$, with the inside of $\gamma$ contained in $D$.
		
		\begin{enumerate}[label=\alph*.]
			\item The number of zeros of $f$ inside $\gamma$ is finite.
			\item $\frac{1}{2 \pi i} \int_{\gamma} \frac{f^{\prime}(z)}{f(z)} dz = N_{Z} - N_{P}$, where $N_{Z}$ is the sum of the orders of the zeros of $f$ inside $\gamma$, and $N_{p}$ is the sum of the orders of the poles of $f$ inside $\gamma$.
		\end{enumerate}
	\end{theorem}
	
	\begin{theorem}
		\emph{(Argument Principle)}
		Suppose $D$ is a domain, $f : D \rightarrow \mathbb{C}$ is non-constant and analytic except at a finite set $P \subset D$ of poles. Let $Z$ be the set of $zeros$ of $f$ in $D$. Assume $\gamma : [a,b] \rightarrow D$ is a piecewise smooth, positively oriented, simple closed curve in $D$ that does not meet $P$ or $Z$, with the inside of $\gamma$ contained in $D$. Suppose that $\alpha(t)$ is a continuous branch of $\text{arg}(f(\gamma(t)))$ that is defined on $[a,b]$. Then
		$$\frac{\alpha(b) - \alpha(a)}{2 \pi} = \frac{1}{2 \pi i} \int_{\gamma} \frac{f^{\prime}(z)}{f(z)} dz = N_{Z} - N_{P}.$$
	\end{theorem}
	
	\begin{theorem}
		\emph{(Rouché)}
		Suppose $f$ and $g$ are analytic on a domain $D$, containing a piecewise smooth simple closed curve $\gamma : [a,b] \rightarrow D$ and its inside. If $|g(z) - f(z)| < |f(z)|$ for every $z \in \gamma$, then, counted with multiplicity, $f$ and $g$ have the same number of zeros inside $\gamma$.
	\end{theorem}
	
	\begin{theorem}
		\emph{(Fundamental Theorem of Algebra)}
		In $\mathbb{C}$, a polynomial of degree $n \geq 1$ has exactly $n$ roots (counted with multiplicity).
	\end{theorem}
	
	\begin{theorem}
		\emph{(Open Mapping Theorem)}
		Suppose $D \subset \mathbb{C}$ is a domain and that $f: D \rightarrow \mathbb{C}$ is analytic. Then either $f(D)$ is a single point, or an open set.
	\end{theorem}
	
	\begin{theorem}
		\emph{(Maximum Principle)}
		Suppose $D \subset \mathbb{C}$ is a domain and $f : D \rightarrow \mathbb{C}$ is analytic and not constant. Then $|f|$ has no local maxima in $D$.
	\end{theorem}
	
	\begin{theorem}
		\emph{(Maximum Principle Version 2)}
		Suppose $D \subset \mathbb{C}$ is a domain and $f : D \rightarrow \mathbb{C}$ is analytic on $D$ and not constant. Suppose also that $D$ is bounded, so that its closure (the union of $D$ and its boundary) is compact, and that $f$ is continuous on the closure of $D$. Then the following are all achieved only on the boundary of $D$:
		\begin{enumerate}[label=\alph*.]
			\item The maximum value of $|f|$.
			\item The maximum and minimum values of $\Re{[f]}$.
			\item The maximum and minimum values of $\Im{[f]}$.
		\end{enumerate}
	\end{theorem}
	
	
\end{document}